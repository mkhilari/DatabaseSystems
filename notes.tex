\documentclass[11pt, a4paper]{article} 

% Packages 
\usepackage[utf8]{inputenc}

% Document Info 
\title{INFO20003 Database Systems}
\author{Manish Khilari}
\date{March 2021}

\begin{document}
    
    \maketitle

    \section{Data and Information}
    \subsection{Data}
    Data is raw facts and figures. 

    \subsection{Information}
    Information is data presented in a context that delivers valuable insights. 

    \subsection{Databases}
    A database is a structured collection of data. \\\\
    Advantages of databases include 
    \begin{itemize}
        \item Data program independence 
        \item Minimal data redundancy (no duplicate data) 
        \item More efficient data sharing 
        \item More efficient data maintenance 
    \end{itemize}

    \subsection{Data Dictionary (Metadata)}
    A data dictionary for a relation (table) is metadata about the attributes 
    (columns) of the relation (table). 
    \begin{center}
        \begin{tabular}{ |c|c|c|c| } 
         \hline
         Column Name & Not Null? & Data Type & Range \\ 
         \hline
        \end{tabular}
    \end{center}
    
    \subsection{Database Management System (DBMS)}
    A DBMS (such as MySQL) enables users to create, access, and update 
    databases. 

    \subsection{Relational Databases}
    \begin{itemize}
        \item Set of relations / entity sets (tables) with entities (rows) and 
        attributes (columns). 
        \item Relation (table) has cardinality (number of rows) and degree / 
        arity (number of columns). 
        \item Relationships are logical links between tables. 
    \end{itemize}

    \subsection{Superkeys (SK)}
    A SK is a set of columns with values that form a tuple that uniquely 
    identifies a row. 
    \begin{itemize}
        \item (ID, FirstName, LastName)
    \end{itemize}

    \subsection{Candidate Keys (CK)}
    A CK is a SK with no smaller SKs as subsets. 
    \begin{itemize}
        \item (ID)
        \item (FirstName, LastName)
    \end{itemize}

    \subsection{Primary Keys (PK)}
    The PK of a table is a chosen CK. All other CKs are alternate keys (AKs). 
    \begin{itemize}
        \item (ID)
    \end{itemize}

    \subsection{Foreign Keys (FK)}
    A FK is a PK of another table used for a relationship. 
    \begin{itemize}
        \item (DogID)
    \end{itemize}
    


    \section{Database Design}
    \textbf{Conceptual $\rightarrow$ Logical $\rightarrow$ Physical} 

    \subsection{Conceptual Design}
    \begin{itemize}
        \item Not database specific 
        \item Entity Relationship (ER) model 
    \end{itemize}

    \subsection{Logical Design}
    \begin{itemize}
        \item Not database specific 
        \item Relational model (table columns, data types) 
    \end{itemize}

    \subsection{Physical Design}
    \begin{itemize}
        \item Database specific 
        \item Implementation details 
    \end{itemize}



    \section{Entity Relationship (ER) Models}
    \subsection{Entities}
    An entity is a uniquely identifiable object with attributes. 

    \subsection{Entity Sets}
    An entity set is a class of entities (same attributes). 

    \subsection{Relationships}
    A relationship is a logical link between entities. 

    \subsection{Relationship Sets}
    A relationship set is a logical link between entity sets. \\\\
    A relationship set can have attributes. 

    \subsection{ER Models}
    \emph{Chen's Notation} \\\\ 

    Employee with ID, FirstName, and LastName \\\\ 
    \begin{itemize}
        \item Entity Employee has attributes (ID, FirstName, LastName) 
        \item (ID) is the PK. 
    \end{itemize}

    Employee with multivalued PhoneNum \\\\ 
    \begin{itemize}
        \item Multivalued (array) 
    \end{itemize}

    Employee with composite Address(Street, Suburb, State) \\\\ 
    \begin{itemize}
        \item Composite (structure) 
    \end{itemize}

    Employee with derived YearsWorked 
    \begin{itemize}
        \item Derived (can be determined using other attributes) 
    \end{itemize}

    Employee work(StartDate) Shop \\\\ 
    \begin{itemize}
        \item Employee has \textbf{zero or many} shops to work 
        \item Shop has \textbf{one or many} employees 
    \end{itemize}

    Employee manage Shop \\\\ 
    \begin{itemize}
        \item Employee has \textbf{zero or many} shops to manage 
        \item Shop has \textbf{exactly one} manager 
    \end{itemize}

    Person own Dog \\\\ 
    \begin{itemize}
        \item Person has \textbf{zero or many} dogs to own 
        \item Dog has \textbf{zero or one} owner 
    \end{itemize}

    \subsection{Weak Entities}
    A weak entity has a FK in its PK. It is in an \textbf{identifying 
    relationship} with its owner. \\\\ 
    \begin{itemize}
        \item Weak entities have \textbf{exactly one} owner (strong entity). 
        \item Weak entities are deleted when their owner is deleted. 
    \end{itemize}
    Strong entity Person(PersonID, Name) has weak entity Child(ChildName, Age) \\\\ 
    \begin{itemize}
        \item (PersonID, ChildName) form PK of Child 
    \end{itemize}



    \section{Relational Models}
    \emph{Crow's Foot Notation} \\\\ 

    Employee manage Shop \\\\ 
    \begin{itemize}
        \item Employee has \textbf{zero or many} shops to manage 
        \item Shop has \textbf{exactly one} manager 
    \end{itemize}
    \begin{verbatim}
        Employee(PK EmployeeID, FirstName NOT NULL, LastName NOT NULL)
        
        Shop(PK ShopID, FK ManagerID NOT NULL, Location NOT NULL)
    \end{verbatim}

    \subsection{Many to Many Relationships (Associative Tables)}
    A many to many relationship is represented by an associative table. \\\\ 
    An associative table has a PK formed by FKs of tables in the many to many 
    relationship. \\\\ 
    Employee work(StartDate) Shop \\\\ 
    \begin{itemize}
        \item Employee has \textbf{zero or many} shops to work 
        \item Shop has \textbf{one or many} employees 
    \end{itemize}
    \begin{verbatim}
        Employee(PK EmployeeID, FirstName NOT NULL, LastName NOT NULL)

        Shop(PK ShopID, Location)

        EmployeeWorksShop(PFK EmployeeID, PFK ShopID, StartDate)
    \end{verbatim}

\end{document}